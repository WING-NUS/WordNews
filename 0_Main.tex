%
%% Based on the style files for EACL 2006 by 
%%e.agirre@ehu.es or Sergi.Balari@uab.es
%% and that of ACL 08 by Joakim Nivre and Noah Smith

\documentclass[11pt]{article}
\usepackage{coling2016}
\usepackage{times}
\usepackage{url}
\usepackage{latexsym}
\usepackage{CJK}
\usepackage{multirow}
\AtBeginDvi{\input{zhwinfonts}}

% 8 pages + 2 pages of references
%\setlength\titlebox{5cm}

% You can expand the titlebox if you need extra space
% to show all the authors. Please do not make the titlebox
% smaller than 5cm (the original size); we will check this
% in the camera-ready version and ask you to change it back.

\title{A Comparison of Word Embeddings for English and Cross-Lingual Chinese Word Sense Disambiguation}
%Muthu: changed title to title case from upper case
%\title{Word Sense Disambiguation Using Word Embeddings}
%Muthu: Isn't this a more accurate title?
%\title{Cross-lingual Word Sense Disambiguation using Word Embeddings}
%\title{A Simplified Approach to Augmenting Word Embeddings to Word Sense Disambiguation}




\author{
Hong Jin Kang$^{1}$, Tao Chen$^{2}$,  Muthu Kumar Chandrasekaran$^{1}$, Min-Yen Kan$^{1,2}$\thanks{ This research is supported by the Singapore National Research Foundation
under its International Research Centre @ Singapore Funding
Initiative and administered by the IDM Programme Office.} \\
$^{1}$School of Computing, National University of Singapore\\
$^{2}$NUS Interactive and Digital Media Institute\\
\\
{\tt  \{kanghongjin\}@gmail.com} \\
{\tt  \{taochen,muthu.chandra,kanmy\}@comp.nus.edu.sg} \\
}

\date{June 2016}

% Tao: As the recent ACL paper~\cite{Iacobacci2016} has evaluated word embeddings for WSD, we have to change our statement, e.g., it is not proper to say "no comparison".


\begin{document}
\maketitle
%\begin{abstract}
 % A key feature of the language-learning application, {\it WordNews}, is to 
 % translate English words to Chinese. The task of Word Sense Disambiguation 
 % (WSD) is the task of identifying the meaning of words in context. We treat 
 % the translation task of WordNews as Cross-Lingual Word Sense Disambiguation, 
 % a variant of WSD.
  
%  To perform Cross-Lingual WSD, we experiment with semi-supervised approaches using word embeddings. 
  % Tao: is IMS state-of-the-art?
  % Hong Jin: it was state-of-the-art just a couple of years ago. Removed mentions of IMS being state-of-the-art
 % We modify an existing WSD system, 
%Muthu: full exopansion when you mention it the first time. 
%`It Makes Sense' (IMS), 
%to make use of word embeddings. 
%We evaluate our approach on the Lexical Sample and All Words tasks in SemEval-2007, Senseval-2, and Senseval-3. We found that word embeddings improves the performance of the existing WSD system. 
%  To evaluate our chosen WSD system on Cross-Lingual WSD, we constructed a publicly available human-annotated English-Chinese evaluation dataset from
%Muthu: is there an unreal variant of new articles?
% Hong JIn: removed
% news articles, and evaluated our system on it.
%Muthu: Did you evaluate the integrated system. If so, mention that result 
%also here
% Hong Jin: mentioned that we evaluated on the CLWSD dataset. I did not carry out an evaluation for the version of WordNews using IMS for WS D
% Finally, we integrated the system into a fork of WordNews.
%\end{abstract}

\begin{abstract}
Word embeddings are now ubiquitous forms of word representation in
natural language processing.  There have been applications of 
%Word Embeddings  
%Muthu: removing title case from 'Word Embeddings' 
word embeddings for monolingual word sense disambiguation (WSD) in English,
but few comparisons have been done.  This paper attempts to bridge
that gap by examining popular embeddings for the task of monolingual
English WSD.  Our simplified method leads to comparable
state-of-the-art performance without expensive retraining.

Cross-Lingual WSD -- where the word senses of a word in a source
language $e$ come from a separate target translation language $f$ --
can also assist in language learning; for example, when providing
translations of target vocabulary for learners.  Thus we have also
applied word embeddings to the novel task of cross-lingual WSD for
Chinese and provide a public dataset for further benchmarking.
%We have also experimented Long Short-term memory neural networks with the same embeddings.
We have also experimented with using word embeddings for LSTM networks
and found surprisingly that a basic LSTM network does not work well.
We discuss the ramifications of this outcome.
\end{abstract}

\section{Introduction}
Learning a new language from language learning websites is 
time consuming. It is, therefore, necessary to make second 
language learning attractive and efficient. Further, since habitual 
learning is effective, we seek to interleave language leaning with 
a popular daily activity. Reading news online is once such 
activity. 
% Muthu: can you find a citation that says lot of people read new online?.
Further recent increase in the popularity of 
portable devices has made online news reading popular than ever 
before~\cite{yarlh2012}. We leverage on this culture to provide 
users of news websites with an opportunity to learn a second 
language.

We propose a system to enable online news readers to efficiently learn 
a new language while they are reading news on news websites. We propose 
a Chrome extension which would run on the client (Chrome web browser) 
when readers visit news websites on a preconfigured list.

% Muthu: can you ealborate a liitle bit? That is, find a citaion that says
% grammar learning, etc., is less important or useful than learning vocabulary (words)
% in a new second langauge 
Learning a new vocabulary is the most time consuming and boring part of 
language learning~\footnote{\url{https://neltachoutari.wordpress.com/tag/vocabulary/}}. 
Perhaps, this justifies the poor adoption of current second language learning 
systems. We, therefore, focus on enabling language learners build their vocabulary 
efficiently while providing them with an enjoyable user experience.

There are many existing language learning software, which, fall into two categories, 
learning by lessons and learning vocabularies. In the first category, 
%learning in lessons, they manually design some lessons to help their 
lessons are purposefully designed to help users easily learn a foreign language.
Duolingo\footnote{\url{https://www.duolingo.com/}} is a popular websites in this category. 
For the second categoryusers are guided to recite lists of words, or provided with 
a translation for their input word in the foreign language. 
Google Translate \footnote{\url{https://translate.google.com/}} stands out in this 
category. The service is available as desktop / mobile / web software including a 
chrome extension. We mainly compare our system  with the aforementioned two software.
Table~\ref{table:difference_summary} summarises important differences between 
our system and all these existing tools. Each difference serves as a motivation 
for developing our extension.

``Duolingo is a free language-learning and crowdsourced text translation 
platform''\footnote{\url{http://en.wikipedia.org/wiki/Duolingo}}.
Most people start to use Duolingo when they know a little or nothing about 
the new language. They starting from some basic lessons and improve step by step.
However, our target audience 
is a mix novice and intermediate level learners of the foreign language. 
We can not only help beginners learn 
a new language but also help them continue their learning by allowing them to practice 
their foreign language. There are also a lot articles with their translations in 
Duolingo, but all the articles and their translations are manually added by 
Duolingo or users from Duolingo. Therefore, parallel articles in Duolingo are old and 
limited. However, our chrome extension is always working even for those up to the 
minute news and our user can just practice their foreign language in their daily 
readings.

\textbf{Google Translate:} ``Highlight or right-click on a section of text and click
on Translate icon next to it to translate it to your 
language''\footnote{\url{http://en.wikipedia.org/wiki/Google_Chrome_Extensions}}. 
% Tao: please cite
Google Translate is a chrome extension that displays only the translation when user 
select a section, which can be a word, a phrase, a sentence or even a whole page. 
Our chrome extension will translate a single word only, and display the translation,
following with the pronunciations and example sentences to help user understand and 
remember this word. Compared with our extension, Google Translate is more like an extension 
to help user understand the content of the page. Furthermore, our extension will display 
the most appropriate translation as it will refer to the context of the word.

%\begin{table}[ht]
%  \caption{Summary of the differences}
%  \label{table:difference_summary}
%  \begin{center}
%  \begin{tabular}{| p{2.4cm} | p{1.2cm} | p{1.2cm} |  p{1.2cm} |}
%    \hline
%    & Duolingo & Google Translate & Chrome Extension \\
%    \hline
%    Lessons & Yes & No & No \\
%    \hline
%    User's foreign language level & Low & Low-High & Low-High \\
%    \hline
%    Time consuming & Yes & No & No\\
%    \hline
%    Resource & Limited & Infinite & Infinite \\
%    \hline
%    Customizable & Yes & No & Yes \\
%    \hline
%    Link to External Dictionary & No & No & Yes \\
%    \hline
%  \end{tabular}
%  \end{center}
%\end{table}
\section{Related Work}

\subsection{Word Sense Disambiguation}


Word Sense Disambiguation is a well-studied problem and there are many different methods. Existing methods can be broadly categorised into supervised approaches, where machine learning techniques are used to learn from labeled training data, and unsupervised knowledge-rich techniques, which do not rely on labeled data. Unsupervised techniques are knowledge-rich, and rely heavily on knowledge bases and thesaurus, such as WordNet. It is noted by Navigli \shortcite{Navigli09wordsense} that supervised approaches using memory-based learning and SVM approaches have worked best. 
%For these approaches, it is common that the only knowledge used is the first sense in WordNet, which is used as a fallback if the system is unable to disambiguate the word in the test data. 

Supervised approaches involve the extraction of features and then classification using machine learning. \shortcite{Zhong2010} developed an open-source WSD system, IMS, which was state-of-the-art at the time it was developed. It is a supervised-learning based WSD system, which first has to be trained using a set of training data. IMS uses three feature types, 1. individual words in the context surrounding the target word, 2. specific ordered sequences of words appearing at specified offsets from the target word, 3. Part-Of-Speech tags of the surrounding 3 words.

Each of the features are binary features, and IMS trains a model for each word. IMS then uses an SVM for classification. IMS is open-source, provides state-of-the-art performance at the time of its publication, and is easy to extend. As such, our work focuses heavily on IMS. 

Training data is required to train IMS, which is a supervised system. 
An example of training data for training WSD system is the One-Million Sense-Tagged Instances \cite{taghipour2015one}. This is the largest dataset we know of for training WSD systems, and we make use of it for training our systems for the All-Words tasks. 

WSD systems can be evaluated using either fine-grained scoring or coarse-grained scoring. In fine-grained scoring, every sense is equally distinct from each other, and answers must exactly match. In coarse-grained scoring, similar senses are grouped and treated as a single sense. A problem of Word Sense Disambiguation is that the granularity of senses are subjective and may not be well-defined. WordNet is a fine-grained resource, and even human annotators have trouble distinguishing between different senses of a word \cite{edmonds2002introduction}. An important measure for any task is the inter-annotator agreement. The inter-annotator agreement is considered the upper bound of a task. 
%In some WSD tasks during Senseval, coarse-grained scoring was done in order to deal with this problem. In these evaluations, similar senses of a word are clustered together and are considered to be the same sense. 


\subsection{Cross-Lingual Word Sense Disambiguation}
Cross-Lingual WSD was partially conceived as a further attempt to solve this issue. In Cross-Lingual WSD, the specificity of a sense is determined by its correct translation. The sense inventory is the possible translations of each word in another language. Two instances are said to have the same sense if they map to the same translation in that language. In SemEval-2010~\cite{Lefever2010}, %~\footnote{\url{http://stel.ub.edu/semeval2010-coref/}}, 
a task for Cross-Lingual WSD was introduced. SemEval-2013~\cite{Lefever2013} %~\footnote{\url{https://www.cs.york.ac.uk/semeval-2013/}} 
featured the second iteration of this task. These tasks were tasks in which an English noun were the targeted words, and the word senses were the translations in Dutch, French, Italian, Spanish and German. 


Traditional WSD approaches are used in Cross-Lingual WSD, although some approaches make use of Statistical Machine Translation methods and features from translation. Cross-Lingual WSD involves training by making use of parallel or multilingual corpora. In the Cross-Lingual WSD task in SemEval-2013, the top approaches used a classification approach or a statistical machine translation approach. 


\subsection{WSD with Word Embeddings}

In NLP, words can be represented in a vector space model. Traditionally, this has been done with {\it one-hot} binary vectors, where there is only one non-zero value in a high-dimensional vector. In this encoding, each dimension represents the presence of a word, and the number of dimensions of the vector space is the size of the vocabulary. In one-hot encoding, all words are considered to be independent of each other. A problem with one-hot encoding is that the large number of dimensions makes machine learning vulnerable to over-fitting. There is no notion of word similarity and all words are independent of each other. A distributed representation of words, such as word embeddings, resolves these problems by encoding words into a low dimensional space. In word embeddings, information about a word is distributed across multiple dimensions, and similar words are expected to be close to each other. Examples of word embeddings are Continuous Bag of Words \cite{mikolovword2vec}, Collobert \& Weston's Embeddings \cite{collobert2008unified}, and GLoVe \cite{pennington2014glove}. We implemented and evaluated the use of word embedding features using these three embeddings in IMS. 


An unsupervised approach using word embeddings for WSD is described by Chen \shortcite{chen2014}. This uses a model for finding representation of senses, rather than just for words, initialised using WordNet's glosses of senses. These sense vectors can then be used during Word Sense Disambiguation. A context vector can be computed by taking the average of the words in a sentence. For disambiguating a single word, the sense with the sense vector that gives maximum Cosine Similarity with this context vector is chosen as the result for disambiguation. Chen {\it et al.} gives an algorithm to disambiguate words starting from the words with fewer senses first. 

A different approach is to work on extending existing WSD systems. Turian \shortcite{Turian10wordrepresentations} suggests that for any existing supervised NLP system, a general way of improving accuracy would be to use unsupervised word representations as additional features. Taghipour \shortcite{Taghipour15} used C\&W embeddings as a starting point and implemented word embeddings as a feature type in IMS. For a specified window, vectors for the surrounding words in the windows, excluding the target word, are obtained from the embeddings and are concatenated, producing $d * (w-1)$ features, where $d$ is the number of dimensions of the vector, and w is the window size. Each feature is a floating point number, which is the value of the vector in a dimension. We note that \cite{Taghipour15} only reported results for C\&W embeddings, and did not experiment on other types of word embeddings.  

Other supervised approaches using word embeddings include AutoExtend \cite{rothe2015autoextend}, which extended word embeddings to create embeddings for synsets and lexemes. In their work, they also extended IMS, but used their own embeddings. Three feature types were introduced by this work, which has some similarities to how Taghipour used word embeddings, but without Taghipour's method of scaling each dimension of the word embeddings. 


Summary:
 - Not many comparison works: \cite{Iacobacci2016}
 
To conclude, word embeddings have been used in several methods to improve on state-of-the-art results in WSD. However, to date, there has been little work investigating how different word embeddings and parameters affect performance of baseline methods of WSD. As far as we know, there has only been one paper comparing the different word embeddings with the use of basic composition methods in WSD. Iacobacci \shortcite{Iacobacci2016} performed an evaluation study of different parameters when enhancing an existing supervised WSD system with word embeddings. Iacobacci noted that the integration of Word2Vec (Skip-gram) with IMS was consistently helpful and provided the best performance. Iacobacci also noted that the composition methods of average and concatenation produced small gains relative to the other composition strategies introduced. However, Iacobacci did not investigate the use of \cite{Taghipour15}'s scaling strategy, which was crucial to improve the performance of IMS.


We also did not find any recent work attempting to integrate modern WSD systems for real-world education usage, and to evaluate the WSD system based on the requirements and suitability for education use. 


\section{Methods}
\label{section:methods}
% Tao: Which classifier did you use?
% Hong Jin: SVM?

As Navigli \shortcite{Navigli09wordsense} noted that supervised approaches have performed best in WSD, we focus on integrating word embeddings in supervised approaches; in specific,
%We explored two approaches for performing supervised WSD using word embeddings. Firstly, we experimented with different methods of composing word embeddings to represent the context of a word and used this in conjunction with IMS, a state-of-the-art supervised WSD system. 
%Secondly, we explored the use of Neural Networks, which have produced state-of-the-art performance in many NLP tasks, in performing WSD. A LSTM network, a type of Recurrent Neural Network, is evaluated. 
we explore the use of word embeddings within the IMS framework. We focus our work on Continuous Bag of Words (CBOW) from Word2Vec,  Global Vectors for Word Representation (GLoVe) and Collobert \& Weston's Embeddings. The CBOW embeddings were trained over Wikipedia, while the publicly available vectors from GloVe and C\&W were used. Word2Vec provides 2 architectures for learning word embeddings, Skip-gram and CBOW. In contrast to Iacobacci \shortcite{Iacobacci2016} which focused on Skip-gram, we focused our work on CBOW.
In our first set of evaluations, we used tasks from Senseval-2, Senseval-3 and SemEval-2007 to evaluate the performance of our classifiers on monolingual WSD. We do this to first validate that our approach is a sound approach of performing WSD, showing improved or identical scores to state-of-the-art systems in most tasks. 

Similar to the work by Taghipour \shortcite{Taghipour15}, we experimented with the use of word embeddings as feature types in IMS. However, we did not just experiment using C\&W embeddings, as different word embeddings are known to vary in quality when evaluated on different tasks \cite{schnabel2015evaluation}. We performed evaluation on several tasks. For the Lexical Sample (LS) tasks of Senseval-2 \cite{senseval2-LS-kilgarriff2001} and Senseval-3 \cite{senseval3-LS-mihalcea2004}, we evaluated our system using fine-grained scoring. For the All Words (AW) tasks, fine-grained scoring is done for Senseval-2 \cite{senseval2-AW-palmer2001} and Senseval-3 \cite{senseval3-AW-snyder2004}; both the fine \cite{semeval2007-fine-pradhan2007} and coarse-grained \cite{semeval2007-coarse-navigli2007} All Words tasks in SemEval-2007 were used. In order to evaluate our features on the All Words task, we trained IMS and the different combinations of features on the One Million Sense-Tagged corpus \cite{taghipour2015one}.

To compose word vectors, one method (used as a baseline) is to sum up
the word vectors of the words in the surrounding context or
sentence. We primarily experimented on this method of composition, due
to its good performance and short training time. For this, every word
vector for every lemma in the sentence (exclusive of the target word)
was summed into a context vector, resulting in $d$ features. Stopwords
and punctuation are discarded. In Turian's
\shortcite{Turian10wordrepresentations} work, two hyperparameters ---
the capacity (number of dimensions) and size of the word embeddings
--- were tuned in his experiments. We follow his protocol and perform
the same in our experiments.

As the remaining features in IMS are binary features, they are not
comparable to the word embeddings which can have unbounded values,
leading to unbalanced influence.  As suggested by Turian
\shortcite{Turian10wordrepresentations}, we should scale down the word
embeddings values, to place their values in the same range as the
other features. The embeddings are scaled to control their standard
deviations. We implement a variant of this technique as done by
Taghipour \shortcite{Taghipour15}, in which we set the target standard
deviation for each dimension. A comparison of different values of the
scaling parameter, $\sigma$ is done. For each $i \in \{1, 2, .. d\}$:
\\

$E_{i} \leftarrow \sigma \times \frac{E_{i}}{stdev(E_{i})} $, where
$\sigma$ is a scaling constant that sets the target standard deviation
\\

We evaluate the effect of varying the scaling factor with the feature
of the sum of the surrounding word vectors. We used word embeddings of 50
dimensions.

\begin{table}[th]
	\caption{Effects of varying scaling factor on C\&W embeddings.}
	\label{table:wordembeddings-accuracy}
	\begin{center}
		\begin{tabular}{| p{7cm} | p{3.5cm} | p{3.5cm} |}
			\hline
			Method & Senseval-2 & Senseval-3 \\
     		 	   &  Accuracy &  Accuracy \\
			\hline
			C\&W, unscaled & 0.569 & 0.641 \\
			\hline
			C\&W, $\sigma _{=0.15}$ & 0.665 & 0.731 \\
			\hline
			C\&W, $\sigma _{=0.1}$ & {\bf0.672} & {\bf0.739} \\
			\hline
			C\&W, $\sigma _{=0.05}$ & 0.664 & 0.735 \\
			\hline
		\end{tabular}
	\end{center}
\end{table}

We experiment with different word embeddings and varied the value of
$\sigma$. Similar to Turian \shortcite{Turian10wordrepresentations}
and Taghipour \shortcite{Taghipour15}, we found that a value of 0.1
for $\sigma$ works well, as seen in Table
\ref{table:wordembeddings-accuracy}. The number of dimensions, known
as the capacity, of the word embeddings was tuned by Turian
\shortcite{Turian10wordrepresentations}. Hence, we varied the values
of the capacity for CBOW and GloVe.  In Tables
\ref{table:wordembeddings-word2vec-accuracy} and
\ref{table:wordembeddings-glove-accuracy} for CBOW and GloVe, the
context sum feature works optimally with 50 dimensions.

In Table \ref{table:top-LS}, we compare the performances of our system
on Senseval-2 (held in 2001) and Senseval-3's (held in 2004) Lexical
Sample tasks with IMS, the top system for each task, and other recent
systems that evaluated on the same task.

\begin{table}[th]
	\caption{Effects of varying capacity on CBOW.}
	\label{table:wordembeddings-word2vec-accuracy}
	\begin{center}
		\begin{tabular}{| p{7cm} | p{3.5cm} | p{3.5cm} |}
			\hline
			Method & Senseval-2 & Senseval-3 \\
     		 	   &  Accuracy &  Accuracy \\
			\hline
			CBOW, $dimensions_{=50}$ & {\bf0.680} & {\bf0.741} \\
			\hline
			CBOW, $dimensions_{=300}$ & 0.669 & 0.731 \\
			\hline
		\end{tabular}
	\end{center}
\end{table}

\begin{table}[th]
	\caption{Effects of varying capacity on GloVe.}
	\label{table:wordembeddings-glove-accuracy}
	\begin{center}
		\begin{tabular}{| p{7cm} | p{3.5cm} | p{3.5cm} |}
			\hline
			Method & Senseval-2 & Senseval-3 \\
        		   & Accuracy & Accuracy \\
			\hline
			GloVe, $dimensions_{=50}$ & {\bf0.678} & {\bf0.741} \\
			\hline
			GloVe, $dimensions_{=100}$ & 0.668 & 0.734 \\
			\hline
			GloVe, $dimensions_{=200}$ & 0.666 & 0.73 \\
			\hline
		\end{tabular}
	\end{center}
\end{table}

Because each dimension is a feature that is used by IMS, if there are
more dimensions, then there are more features. This may result in
overfitting on small datasets. This is a possible reason that the
smaller number of dimensions work better.

%Muthu: table seems to exceed the margin. attempting to fix
\begin{table}[th]
	\caption{Comparison of systems on Lexical Sample tasks. Rank 1 system refers to the top system for the specified task during the evaluation.}
	\label{table:top-LS}
	\begin{center}
		\begin{tabular}{| p{7cm} | p{3.5cm} | p{3.5cm} |}
			\hline
			Method & Senseval-2 LS & Senseval-3 LS \\
            & Accuracy & Accuracy \\
			\hline
			IMS + CBOW $\sigma _{=0.1}$ (proposed) & 0.680 & 0.741 \\
			\hline
            IMS + CBOW $\sigma _{=0.15}$ (proposed) & 0.670 & 0.734 \\
			\hline
			
			IMS & 0.653 & 0.726\\
			\hline
			Rank 1 System & 0.642 \cite{florian2002combining} & 0.729 \cite{grozea2004finding} \\
            %Muthu: cite the paper of this rank 1 system.
            % Hong Jin: added
			\hline
			\newcite{rothe2015autoextend} & 0.666 & 0.736 \\
			\hline
			\newcite{Taghipour15} & 0.662 & 0.734 \\
			\hline
           	IMS + Word2Vec (Skip-gram) \shortcite{Iacobacci2016}  & {\bf0.699} & {\bf0.752} \\
			\hline
			Most Frequent Sense (Baseline) & 0.476 & 0.552 \\
			\hline
		\end{tabular}
	\end{center}
\end{table}

Apart from the Lexical Sample tasks, we evaluated our systems on the All Words tasks of Senseval-2, Senseval-3 and SemEval-2007, which were evaluated on by Zhong and Ng \shortcite{Zhong2010}. As seen in Table \ref{table:All-AW}, our enhancements to IMS to make use of word embeddings give better results on All Words task than the original IMS and the respective Rank~1 systems from the original tasks. It also outperforms several recent systems developed and evaluated in recent papers. We note that although our system increased accuracy on IMS on several
% Tao: which test did you use? 
% Hong Jin: mcnemar's test
% Min: edited in
All Words tasks, the differences were not statistically significant (as measured
using McNemar's test for paired nominal data).

%Muthu: table seems to exceed the margin. you could wrap the headers. Fixing.
\begin{table}[th]
	\caption{Accuracy of our system on Senseval-2, Senseval-3, SemEval-2007 All Words task.}
	\label{table:All-AW}
	\begin{center}
		\begin{tabular}{| p{4cm} | p{2cm} | p{2cm} | p{2.5cm} | p{2.5cm} | }
			\hline
			Method & Senseval-2 & Senseval-3 & SemEval-2007 & 
            SemEval-2007 \\
			& AW Accuracy & AW Accuracy & Fine-Grained &
			Coarse-grained \\
			\hline
			IMS + CBOW, 
			
			$\sigma _{=0.1}$ (proposed) & 0.677 & 0.679 & 0.604 & 0.826  \\
			\hline
            IMS + 
			
			CBOW, $\sigma _{=0.15}$ (proposed) & 0.673 & 0.675 & {\bf0.615} & {\bf 0.828 } \\
			\hline
			
			\cite{Taghipour15} & -& {\bf0.682} & - & - \\
			\hline
			\cite{chen2014} & - & - & - & 0.826  \\
			\hline
			IMS (on One Million Sense-Tagged dataset) & 0.682 & 0.674 & 0.585 & 0.816 \\
            \hline
            IMS + Skip-gram \cite{Iacobacci2016}  & {\bf0.683} & {\bf0.682} & 0.591 & - \\
			\hline
			IMS (original) & 0.682 & 0.676 & 0.583 & 0.826   \\
			\hline
			Top System during the task & {\bf0.69} & 0.652 & 0.591 & 0.825  \\
			\hline
			WordNet Sense 1 & 0.619 & 0.624 & 0.514 & 0.789\\
			\hline
		\end{tabular}
	\end{center}
\end{table}


It can be seen in Table \ref{table:top-LS} and \ref{table:All-AW} that the simple enhancement of integrating word embedding using the baseline composition method, followed by the scaling step, improves the existing IMS system, and we get performance comparable to or better than top approaches in both Lexical Sample tasks and All Words tasks. 

%Muthu: if we have significance tests for these we should add them to the table
% for e.g., add a ** to denote $p < 0.01$ and * to denote $p < 0.05$
%Muthu: this table is also exceeding margin. fixing
%Muthu: Use multirow to merge rows for the columns where the text repeats. This table looks 
% cluttered
% Muthu: Is it okay to reorg the table such that all the same Types are together? I can do it if okay.
% Hong Jin: I started some work here to use multirow, but it will be great if you could continue to reorg the table. 
\begin{table}[th]
	\caption{Accuracy of adding word embeddings to IMS on Senseval-2, Senseval-3 Lexical Sample and All Words tasks and SemEval-2007 All Words task}
\vspace{0.15cm}
	\label{table:full}
\centering
\begin{tabular}
%{|p{1cm}|p{0.5cm}|p{1cm}|p{1.5cm}|p{1.5cm}|p{1.5cm}|p{1.5cm}|p{1.5cm}|p{1.5cm}|p{1.5cm}|}
{|l|r|r|r|r|r|r|r|r|r|}
\hline
Type & Size & Compose & Scaling & SE-2 & SE-3 & SE-2 & SE-3 & SE-2007 & SE-2007 \\
 	&  &  &  &  LS &  LS & AW & AW &  Fine- & Coarse- \\
   	&  &  &  &	   &     &    &    &  grained & grained \\
\hline
\multirow{3}{*}{C\&W}&\multirow{3}{*}{50}&\multirow{3}{*}{Sum}&0.05&0.666&0.734&0.679&0.673&0.594&0.818 \\

 & & &0.1&0.671&0.738&0.678&0.673&0.6&0.819 \\

 & & &0.15&0.666&0.732&0.675&0.672&0.598&0.817 \\
\hline
\multirow{3}{*}{CBOW}&\multirow{3}{*}{50}&\multirow{3}{*}{Sum}&0.05&0.672&{\bf 0.744}&{\bf 0.68}&0.677&0.604&0.824\\

&&&0.1&{\bf 0.68}&0.741&0.677&0.679 &0.604 & 0.826\\

&&&0.15&0.67&0.734&0.673&0.675&{\bf 0.615}&{\bf 0.828}\\
\hline
\multirow{3}{*}{Glove}&\multirow{3}{*}{50}&\multirow{3}{*}{Sum}&0.05&0.675&0.738&0.676&0.678&0.596&0.819 \\

& & &0.1&0.679&0.741&0.678&0.68&0.594&0.819 \\

& & &0.15&0.674&0.731&{\bf 0.68}&0.678&0.591&0.819 \\
\hline
\multirow{3}{*}{CBOW}&\multirow{3}{*}{200}&\multirow{3}{*}{Sum}&0.05&0.679&0.742&0.679&0.68&0.602&0.823 \\

& & &0.1&0.669&0.731&0.676&0.675&0.602&0.82 \\

& & &0.15&0.651&0.715&0.667&0.673&0.594&0.822 \\
\hline
\multirow{3}{*}{Glove}&\multirow{3}{*}{200}&\multirow{3}{*}{Sum}&0.05&0.682&0.741&0.68&{\bf0.682}&0.6&0.823 \\

& & &0.1&0.666&0.73&0.677&0.679&0.591&0.827 \\

& & &0.15&0.654&0.706&0.674&0.675&0.591&0.826 \\
\hline
C\&W&50&Concat&0.1&0.659&0.724&0.679&0.674&0.585&0.818 \\
\hline
\multirow{2}{*}{CBOW}&50&\multirow{2}{*}{Concat}&0.1&0.66&0.725&0.678&0.672&0.581&0.816\\

&200&&0.1&0.667&0.729&0.675&0.67&0.591&0.819\\
\hline
\multirow{2}{*}{Glove}&50&\multirow{2}{*}{Concat}&0.1&0.657&0.722&0.679&0.671&0.583&0.818\\

&200& &0.1&0.664&0.728&0.677&0.669&0.587&0.817\\
\hline
\end{tabular}
\end{table}

We note that a smaller size of the word embeddings generally improved
performance on the Lexical Sample task, however, this effect was not
observed in the All Words task. We also note that relatively poorer
performance in the Lexical Sample tasks may not necessarily result in
poor performance on the All Words task. We see from the results that
the combination of \cite{Taghipour15}'s scaling strategy and summation
produced results better than the proposal in \cite{Iacobacci2016} to
concatenate and average (0.651 and 0.654), suggesting that the scaling
factor is important for the integration of word embeddings for
supervised WSD.

%\iffalse
\subsection{LSTM Network}

A Long Short Term Memory (LSTM) network is a type of Recurrent Neural
Network which has recently been shown to have good performance on many
NLP classification tasks. Unlike normal neural networks which can only
accept a fixed size vector as an input, recurrent neural networks
accept variable sized inputs. As such, recurrent neural networks can
operate over sequences of word vectors and perform operations on them
sequentially. The potential benefit of this approach over our existing
approach in IMS is this is that the neural network is able to use
information about the sequence of words in classification. Examples of
using a neural network for classification are
\cite{socher2011parsing} and \cite{socher2013recursive}.  
% Min: how well did they do?  Comparable to yours?
% Hong Jin: Yuan did not use the same test dataset as us. kaageback only reported results on the lexical sample tasks. added to the table below. I also removed Yuan since the final classifier used was not a LSTM
For WSD, K{\aa}geb{\"a}ck \cite{kaageback2016word} explored the direct use of bidirectional LSTMs, along with the use of dropwords. In our approach, we explore a simpler
na\"{\i}ve approach instead.

For the Lexical Sample tasks, we train the model on the training data
provided for the task. For the All Words task, we trained the model on
the One Million Sense-Tagged dataset. For each task, similar to IMS,
we train a model for each word, using GloVe word embeddings as the input layer.

\begin{table}[th]
	\caption{Accuracy of a basic LSTM approach on the Lexical
          Sample tasks.}
	\label{table:NN-LS}
	\begin{center}
		\begin{tabular}{| p{6cm} | p{4cm} | p{4cm} |}
			\hline
			Method & Senseval-2 Accuracy & Senseval-3 Accuracy \\
			\hline
			LSTM approach (Proposed) & 0.458  & 0.603 \\
			\hline
			IMS & 0.653 & 0.726\\
            \hline
            \cite{kaageback2016word} & 0.669 & 0.734 \\
			\hline
			Rank 1 System during the task & 0.642 & 0.729 \\
			\hline
			Most Frequent Sense (Baseline) & 0.476 & 0.552 \\
			\hline
		\end{tabular}
	\end{center}
\end{table}

\begin{table}[th]
	\caption{Accuracy of a basic LSTM approach on the All Words tasks.}
	\label{table:NN_AW}
	\begin{center}
		\begin{tabular}{| p{7cm} | p{2cm} | p{2cm} | p{2cm} | }
			\hline
			Method & Senseval-2 Accuracy & Senseval-3 Accuracy\\
			\hline
			LSTM approach (Proposed) & 0.619  & 0.623  \\
			
			\hline
			IMS (trained on One Million Sense-Tagged dataset) & 0.682 & {\bf0.674} \\
			\hline
			Rank 1 System during the task & {\bf0.69} & 0.652  \\
			\hline
			Wordnet Sense 1 & 0.619 & 0.624  \\
			\hline
		\end{tabular}
	\end{center}
\end{table}

The performance of the na\"{\i}ve LSTM is poor in both type of tasks (Tables
\ref{table:NN-LS} and \ref{table:NN_AW}). The models converge to just
using the most common sense for the All Words task. 
% Min: Repetitive -- should have some examples to show this for an
% individual word or two, or to show that words that have more
% instances have markedly better performance.

% Hong Jin: removed repeated paragraph
% Hong JIn: Also removed the reason since it may not make sense. If it were true that that is the main reason for the poor performance, then the other paper using LSTM should have the same problem. I will continue to try to figure out why this problem happens. In the meantime, I have cited some papers to try to better support this argument

A possible reason for this is overfitting. WSD suffers
from the problem of data sparsity. Although there are many training
instances in total, as we train a separate model for each word, there are training examples for each
individual word. Taghipour \shortcite{Taghipour15} indicated the need to prevent overfitting while using a neural network to adapt C\&W embeddings by omitting a hidden layer and adding a Dropout layer, while K{\aa}geb{\"a}ck \shortcite{kaageback2016word} developed a novel regularization technique, dropword.

\section{English-Chinese Cross-Lingual Word Sense Disambiguation}
\label{section:CLWSD}

% Tao: papers about CLWSD

% sem-eval clwsd task: \cite{Lefever2010}, \cite{Lefever2013}
% cross-lingual embeddings: \cite{Shi2015}, \cite{Coulmance2015}, \cite{Aldarmaki2016}
% build word embeddings for WSD: \cite{Guo2014}
% clwsd dataset: \cite{Rekabsaz2016} (English-Persian)

% survey on wsd: \cite{Navigli2009}


We then evaluate our proposal on Cross-Lingual Word Sense Disambiguation task.
One key application of such task is to facilitate language learning systems.  
For example, {\it MindTheWord}\footnote{\url{https://chrome.google.com/webstore/detail/mindtheword/fabjlaokbhaoehejcoblhahcekmogbom}} and {\it WordNews}~\cite{tao2014} are two applications that allow users to learn vocabulary of a second language in context, in the form of providing translations of words in an online article.
%which is required for effective acquisition of vocabulary \cite{Hirsch03readingcomprehension}. These applications often rely on translation systems to provide translations. 
In this work, we model this problem of finding translations of words as a variant of WSD, Cross-Lingual Word Sense Disambiguation, which was also taken in \cite{tao2014}.
% Tao: stop here. 

In the earlier section, we have validated and compared enhancements to IMS using word embeddings. These have produced results comparable to, and in some cases, better than state-of-the-art performance on the monolingual WSD tasks. We further evaluate our approach for use in the Cross-Lingual Word Sense Disambiguation for performing contextually-appropriate translations of single words. To accomplish this, we first construct a English-Chinese Cross-Lingual WSD dataset. For our sense inventory, we work with the existing dictionary in the open-source educational application, WordNews \cite{tao2014}, which contains a dictionary of English words and their possible Chinese translations. After which, we integrate the trained system into a fork of WordNews. 

\subsection{Dataset}
In order to construct an evaluation dataset, we hired human annotators and constructed a English-Chinese Cross-Lingual WSD dataset using sentences from recent news articles. As far as we know, there is no existing publicly available English-Chinese Cross-Lingual WSD dataset. As the dataset is constructed using recent news data, it is a good representation for the use case in WordNews. {\footnote{The dataset can be obtained at %{\url{  https://kanghj.github.io/eng_chinese_news_clwsd_dataset/}}}}
{\url{https://to.be.made.public/if/accepted}}}

To obtain the gold standard for this data set, we hired 18 annotators to select the right translations for a given word and its context. There are 697 instances in total in our dataset, with a total of 251 target words to disambiguate. There are about 116 instances (15 annotators with 116 instances, 3 with 117) for each individual to annotate physically on hard-copy spreadsheet. Each instance will be annotated by 3 different annotators. The annotators are all bilingual undergraduate students, who are native Chinese speakers. 

For each instance, which contains a single English target word to disambiguate, we include the sentence it appears in and its adjacent sentences as its context. Each instance contains possible translations of the word. The sense inventory is from our dictionary of English-Chinese pairs, crawled from Google Translate and Bing Translator. The annotators will select every Chinese word that has an identical meaning to the English target word. We instructed that, if the word cannot be translated appropriately, the annotators should leave it blank. The participants can provide their own translations to the word if they believe that there is a suitable translation, but was not provided. 


In WSD, it is important to obtain the inter-annotator agreement of the dataset. The concept of a sense is a human construct, and therefore, as earlier elaborated on when discussing sense granularity, it is subjective and may be difficult for human annotators to agree on the correct answer. We try to measure the inter-annotator agreement using pairwise Cohen's Kappa. Our annotation task differs from the usual since we allow users to select multiple labels for each case. In addition, annotators can also add new labels to the case if they do not agree with any label provided. As such, applying the Cohen's Kappa as it is does not work for our annotated dataset. We note that some work has been done for multi-label Kappa, such as by Rosenberg \shortcite{rosenberg2004augmenting}, however the situation described  is different from our case, as we cannot assume a uniform distribution of labels or that there is a primary label among the multiple labels selected by an annotator. We are also unable to compute the probably of chance agreement by word, since there are few test instances per word in our dataset.

The Kappa equation is given as 
$\kappa = \frac{p_A - p_E}{1 - p_E} $.
To compute $p_A$ for $\kappa$, we use a simplified, optimistic approach where we select one annotated label out of possibly multiple selected labels for each annotator. We always choose the label that results in an agreement between the pair, if such a label exist. For $p_E$, the probability of chance agreement, as the labels of each case are different, we consider the labels in terms of how frequent they occur in the training data. We only consider the top 3 most frequent senses for each word, and ignore the other labels due to a skewed sense distribution. We first compute the probability of an annotator selecting each of the top three frequent senses, $p_E$ is then equals to the sum of the probability that both annotators selected one of the three top senses by chance. 

We present the probabilities that an annotator will select each of the top three senses in Table \ref{table:IAA}. The value of $p_E$ by this proposed method of computation is 0.186. The pairwise value of $\kappa$ is obtained is 0.42. We interpreted this as a moderate level of agreement. We note that the number of possible labels that can be assigned to each case is large, which is known to affect the value of $\kappa$ negatively. This is worsened since we allow the annotators to add new labels. 

\begin{table}[ht]
	\caption{Probability of an annotator annotating the top three senses}
	\label{table:IAA}
	\begin{center}
		\begin{tabular}{| p{4cm} | p{4cm}  | p{4cm} | }
			\hline
			Most Frequent Sense & 2nd Most Frequent Sense & 3rd Most Frequent Senses\\
			\hline
			0.343 & 0.206 & 0.161\\						
			
			\hline
		\end{tabular}
	\end{center}
\end{table}

As we consider any overlap in annotated labels to be a match, this approach may overestimate the agreement between annotators. However, in our dataset, a significant number of annotators (5 out of 18) only selected a single translation in the dataset instead of every suitable translation. As in this annotation task, we consider the possible translations as  fine-grained, the value of agreement is likely to be underestimated in this case. Hence, we believe that clustering of similar translations during annotation is required in order to deal with the issue of sense granularity in Cross-Lingual WSD. To overcome this problem, we used different configurations of granularity during evaluation of our system. In the most relaxed configuration, we assume that all annotations by the annotators are correct answers. On the strictest configuration, all three annotators must agree on the translation before it is considered to be the correct sense. For all configurations, we remove instances from the dataset if it does not have a correct sense. We excluded instances with out-of-vocabulary annotations (added by the annotators if they did not think any of the provided translations are suitable) were excluded from the test set.


For the first configuration, we included all instances annotated by the participants. For the second configuration, we omitted bad instances and only consider a translation to be correct if more than one participant agreed on that translation. For the third configuration, we included only answers where all three participants agreed on the answer. We also noticed that some target words were part of a proper noun, such as the word 'white' in 'White House'. This led to some confusion among annotators, so we omitted instances where the target word is part of a proper noun. Statistics of the test dataset after filtering out the above cases are given in Table \ref{table:CLWSD-test-stats-no-ne}.


\begin{table}[ht]
	\caption{Statistics of our dataset}
	\label{table:CLWSD-test-stats-no-ne}
	\begin{center}
		\begin{tabular}{| p{8cm} | r| r|}
			\hline
			Configuration & \# of instances & \# of unique target words \\
			\hline
			Include all & 653 & 251\\ 
			\hline
			Exclude instances with OOV annotations & 481 & 206 \\						
			\hline
			Exclude instances without at least partial agreement & 412 & 193 \\
			\hline
			Exclude instances without complete agreement & 229 & 136 \\
			\hline
		\end{tabular}
	\end{center}
\end{table}

\subsection{Experiments}

As IMS is a supervised system, we need training data before we can use it. We constructed data by processing a parallel corpus, the news section of the UM-Corpus \cite{tian2014corpus}, and performing word alignment. We used the dictionary provided by \cite{tao2014} as the sense inventory. At the start of this work, we further expanded the dictionary using translations from Bing Translator and Google Translate. For construction of the training dataset, word alignment is used to assign Chinese words as training labels for each English target word. GIZA++ \cite {och03} is used for word alignment. To evaluate our system, we compare the results of the method described in \cite{tao2014}, which uses Bing Translator and word alignment to obtain translations.


\begin{table}[ht]
	\caption{Results of our systems on the Cross-Lingual WSD dataset, without named entities. Instances with out-of-vocabulary annotations are removed. All annotations are considered correct answers.}
	\label{table:CLWSD-test-results}
	\begin{center}

			\begin{tabular}{| p{9cm}| r| }
				\hline
				Method & Accuracy \\
				\hline
				Bing Translator + 
                word alignment (baseline) & 0.559  \\
				\hline
				IMS & 0.752  \\
				\hline
                IMS + CBOW, 50 dimensions, $\sigma _{=0.05}$ (proposed) &  0.763  \\
				\hline
				IMS + CBOW, 50 dimensions, $\sigma _{=0.1}$ (proposed) & {\bf 0.772}  \\                                
                \hline
                IMS + CBOW, 50 dimensions, $\sigma _{=0.15}$ (proposed) & 0.767  \\
                \hline
                IMS + GLoVe, 50 dimensions, $\sigma _{=0.1}$ (proposed) & 0.761  \\


				\hline
			\end{tabular}

	\end{center}
\end{table}

We use the configuration where every annotation is considered to be correct for our main evaluation since this is closer to a coarse-grained evaluation. 

It can be seen that word embeddings improves the performance on Cross-Lingual WSD. Similar to our observations for monolingual WSD, the use of both CBOW and GLoVe improved performance. However, the improvements from the word embeddings feature type over IMS was not statistically significant at 95\% confidence level. This is attributed to the small size of the dataset. 


\subsection{Reason that Bing Translator is evaluated poorly in our experiments}
Bing Translator scores poorly in our evaluation with our annotated dataset as seen in Table \ref{table:CLWSD-test-results}. This could be because Bing Translator performs translation at the phrase-level. Therefore, many of the target words are not translated individual and is translated only as part of a larger unit, making it less suitable for the use case in WordNews where only the translation of the single word matters. For example, when traslating the word `'`little'' in ``... little kids ...'', the word alignment information in Bing Translator does not give an alignment for the word `little' but instead translates the entire multi-word unit ``little kids''. 
%Since the gold standard was produced by annotations before we ran the experiments, none of the %participants would indicate that the translation of ``little kids'' is the translation for %``little''. 
As such, the translation would not match any of the annotations provided by our annotators. This is an appropriate treatment since a user of an educational app requesting a translation for the word `little' should not see the translation of the entire phrase.

\section{Conclusion}
\label{section:conclusion}

After we have evaluated the performance of the systems on the this
Cross-Lingual WSD dataset, we integrate the top-performing system
using word embeddings and the trained models into a fork of the
WordNews system. We experimented and implemented with different
methods of using word embeddings for supervised WSD. We tried two
approaches, by enhancing an existing WSD system, IMS, and by trying a
neural approach using a simple LSTM.  We evaluated our apporach as
well as various methods in WSD, against initial evaluations on the
existing test data sets from Senseval-2, Senseval-3, SemEval-2007. In
a nutshell, adding any pretrained word embedding as a feature type to
IMS resulted in the system performing competitively or better than the
state-of-the-art systems on many of the tasks. This supports
\cite{Iacobacci2016}'s conclusion that concluded that existing
supervised approaches can be augmented with word embeddings to give
better results.

Our findings also validated Iacobacci\shortcite{Iacobacci2016}'s
findings that Word2Vec gave the best performance. However, we also
note that, other than Word2Vec, other publicly available word
embeddings, Collobert \& Weston's embeddings and GLoVe also
consistently enhanced the performance of IMS using the summation
feature with little effort. Other than on the Lexical Sample tasks,
where smaller word embeddings performed better, we also found that the
number of dimensions did not affect results as much as the scaling
parameter. Unlike Iacobacci et al. \shortcite{Iacobacci2016}, we also
found that a simple composition method using summation already gave
good improvements over the standard WSD features, provided that the
scaling method described in \cite{Taghipour15} was performed.

An additional key contribution of our work was to build a
gold-standard English-Chinese Cross-Lingual WSD dataset constructed
with sentences from real news articles and to evaluate our proposed
word embedding approach under this scenario.  Our compiled dataset was
used as evaluation of the task of translating English words on online
news articles. This dataset is made available publicly.  We observed
that word embeddings also improves the performance of WSD in our
Cross-Lingual WSD setting.

As future work, we will examine how to expand the existing dictionary
with more English words of varying difficulty and include more
possible Chinese translations, as we note that there were several
instances in the Cross-Lingual WSD dataset where the annotators did
not choose an existing translation. An extrinsic evaluation of the
Cross-Lingual WSD system should be done with users of different
language learning application in order to validate that the quality of
translations did indeed improve in real world usage.


\bibliographystyle{acl}
\bibliography{socreport}
\end{document}
