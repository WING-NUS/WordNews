\section{Related Work}
There are many existing language learning software, which, fall into two categories, learning by  
lessons and learning vocabularies. In the first category, %learning in lessons, they manually design some lessons to help their 
lessons are purposefully designed to help user learn a foreign language in an easier way.
Duolingo\footnote{\url{https://www.duolingo.com/}} is a popular websites in this category. 
For the second category
%, learning follow vocabularies, 
% Tao: "they" refers to who? Same problem for the 1st category. Edited both for you 
% they let user recite words' list or give the corresponding translations to user's input, 
users are guided to recite lists of words, or provided with a translation for their input word 
in the foreign language.
%a corresponding translation is given according to user's input.
Google Translate \footnote{\url{https://translate.google.com/}} stands out in this category. The service 
is available as desktop / mobile / web software including a chrome extension. Our chrome extension is 
%Muthu: say why
different from all these existing tools. 
%Muthu: what are the diference? have you already provided the list?
I mainly compare our system  with the aforementioned two softwares. Each difference serves as a motivation 
for developing our extension.

``Duolingo is a free language-learning and crowdsourced text translation platform''\footnote{\url{http://en.wikipedia.org/wiki/Duolingo}}.
Most people start to use Duolingo when they know a little or nothing about the new language. They starting 
from some basic lessons and improve step by step.
% Tao: your comparison is good, but you need to highlight the uniquencess of your extension in introduction or some other relevant section. 
However, our target audience include people who know nothing about the foreign language language as well as people with a who are 
fairly in the foreign language. We can not only help beginners learn a new language but also help them continue their learning 
by allowing them to practice their foreign language. There are also a lot articles with their translations in Duolingo, but 
all the articles and their translations are manually added by Duolingo or users from Duolingo. Therefore, parallel articles in Duolingo 
are old and limited. However, our chrome extension is always working even for those up to the minute news and our user can just practice 
their foreign language in their daily readings.

\textbf{Google Translate:} ``Highlight or right-click on a section of text and click on Translate icon next to it to translate it to your 
language"\footnote{\url{http://en.wikipedia.org/wiki/Google_Chrome_Extensions}}. % Tao: please cite
Google Translate is a chrome extension that displays only the translation when user select a section, which can be a word, a phrase, a 
sentence or even a whole page. Our chrome extension will translate a single word only, and display the translation, following with the 
pronunciations and example sentences to help user understand and remember this word. Compared with our extension, Google Translate is more 
like an extension to help user understand the content of the page. Furthermore, our extension will display the most appropriate translation 
as it will refer to the context of the word.
% Tao: Not accurate. If selecting a whole sentence, Google translator also uses context. You may change to "selecting for a single word"?

As far as I know, Google Translate will not refer to the context while selecting a single word as the selected section is the only input.
\\
\begin{table}[ht]
  \caption{Summary of the differences}
  \label{table:difference_summary}
  \begin{center}
  \begin{tabular}{| p{2.4cm} | p{1.2cm} | p{1.2cm} |  p{1.2cm} |}
    \hline
    & Duolingo & Google Translate & Chrome Extension \\
    \hline
    Lessons & Yes & No & No \\
    \hline
    User's foreign language level & Low & Low-High & Low-High \\
    \hline
    Time consuming & Yes & No & No\\
    \hline
    Resource & Limited & Infinite & Infinite \\
    \hline
    Customizable & Yes & No & Yes \\
    \hline
    Link to External Dictionary & No & No & Yes \\
    \hline
  \end{tabular}
  \end{center}
\end{table}
\\