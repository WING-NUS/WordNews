%
% File acl2015.tex
%
% Contact: car@ir.hit.edu.cn, gdzhou@suda.edu.cn
%%
%% Based on the style files for ACL-2014, which were, in turn,
%% Based on the style files for ACL-2013, which were, in turn,
%% Based on the style files for ACL-2012, which were, in turn,
%% based on the style files for ACL-2011, which were, in turn, 
%% based on the style files for ACL-2010, which were, in turn, 
%% based on the style files for ACL-IJCNLP-2009, which were, in turn,
%% based on the style files for EACL-2009 and IJCNLP-2008...

%% Based on the style files for EACL 2006 by 
%%e.agirre@ehu.es or Sergi.Balari@uab.es
%% and that of ACL 08 by Joakim Nivre and Noah Smith

\documentclass[11pt]{article}
\usepackage{acl2015}
\usepackage{times}
\usepackage{url}
\usepackage{latexsym}
\usepackage{algorithm}
\usepackage{algorithmic}
\usepackage{fullpage}
\usepackage{graphicx}
\usepackage{longtable}
\usepackage{lscape}
\usepackage{CJKutf8}
\usepackage[hyphens]{url}
%\begin{CJK*}{UTF8}{gbsn}
%\setlength\titlebox{5cm}

% You can expand the titlebox if you need extra space
% to show all the authors. Please do not make the titlebox
% smaller than 5cm (the original size); we will check this
% in the camera-ready version and ask you to change it back.

% Tao: All upper-case is hard to read.
%\title{LEARNING SECOND LANGUAGE FROM NEWS WEBSITES}
\title{Learning Second Languages from News Websites}

%\author{First Author \\
%  Affiliation / Address line 1 \\
%  Affiliation / Address line 2 \\
%  Affiliation / Address line 3 \\
%  {\tt email@domain} \\\And
%  Second Author \\
%  Affiliation / Address line 1 \\
%  Affiliation / Address line 2 \\
%  Affiliation / Address line 3 \\
%  {\tt email@domain} \\}

\date{}

\begin{document}

\maketitle
\begin{abstract}
%Learning a second language is difficult
%and requires constant revision and immersion.
%Fortunately, many of us read news
%online everyday. In this paper, we propose
%a web browser extension that allows
%a reader to learn a second language vocabulary
%while reading news online. We hypothesize
%Since we find a word’s context to be useful
%in learning a vocabulary, we further
%use word sense disambiguation (WSD) to
%show the best translation for each word in
%the context. Our proposed WSD method,
%leveraging the extension of standard machine
%translation system, significantly betters
%baseline methods in both coverage
%and accuracy. We also elaborate on
%the issues of determining appropriate distractors
%for multiple-choice word mastery
%quizzes .
%We conducted a user survey to evaluate
%our system against user requirements collected
%through an earlier survey.


Learning a second language is difficult and requires
constant revision and immersion.  Fortunately, many of us 
read news online everyday. In this paper, we propose a web browser extension that allows readers to learn a second language vocabulary 
while reading news online. To this end, we propose algorithms to disambiguate  word sense and translate the words in new articles properly 
to target langauge.
%for translating source language words in news articles to target 
%language. We find that a standard machine translation system extended 
%with a word sense disambiguation (WSD) system significantly betters baseline translation methods in both coverage and accuracy.
We find a machine translation based method significantly betters baselines in both coverage and accuracy.
We also propose techniques for generating appropriate distractors for multiple-choice word mastery quizzes for assessing language learners.
We conducted a user survey to evaluate our system.
%% complete what is your finding and what we learn from it

\end{abstract}

%\section{Introduction}
Learning a new language from language learning websites is 
time consuming. It is, therefore, necessary to make second 
language learning attractive and efficient. Further, since habitual 
learning is effective, we seek to interleave language leaning with 
a popular daily activity. Reading news online is once such 
activity. 
% Muthu: can you find a citation that says lot of people read new online?.
Further recent increase in the popularity of 
portable devices has made online news reading popular than ever 
before~\cite{yarlh2012}. We leverage on this culture to provide 
users of news websites with an opportunity to learn a second 
language.

We propose a system to enable online news readers to efficiently learn 
a new language while they are reading news on news websites. We propose 
a Chrome extension which would run on the client (Chrome web browser) 
when readers visit news websites on a preconfigured list.

% Muthu: can you ealborate a liitle bit? That is, find a citaion that says
% grammar learning, etc., is less important or useful than learning vocabulary (words)
% in a new second langauge 
Learning a new vocabulary is the most time consuming and boring part of 
language learning~\footnote{\url{https://neltachoutari.wordpress.com/tag/vocabulary/}}. 
Perhaps, this justifies the poor adoption of current second language learning 
systems. We, therefore, focus on enabling language learners build their vocabulary 
efficiently while providing them with an enjoyable user experience.

There are many existing language learning software, which, fall into two categories, 
learning by lessons and learning vocabularies. In the first category, 
%learning in lessons, they manually design some lessons to help their 
lessons are purposefully designed to help users easily learn a foreign language.
Duolingo\footnote{\url{https://www.duolingo.com/}} is a popular websites in this category. 
For the second categoryusers are guided to recite lists of words, or provided with 
a translation for their input word in the foreign language. 
Google Translate \footnote{\url{https://translate.google.com/}} stands out in this 
category. The service is available as desktop / mobile / web software including a 
chrome extension. We mainly compare our system  with the aforementioned two software.
Table~\ref{table:difference_summary} summarises important differences between 
our system and all these existing tools. Each difference serves as a motivation 
for developing our extension.

``Duolingo is a free language-learning and crowdsourced text translation 
platform''\footnote{\url{http://en.wikipedia.org/wiki/Duolingo}}.
Most people start to use Duolingo when they know a little or nothing about 
the new language. They starting from some basic lessons and improve step by step.
However, our target audience 
is a mix novice and intermediate level learners of the foreign language. 
We can not only help beginners learn 
a new language but also help them continue their learning by allowing them to practice 
their foreign language. There are also a lot articles with their translations in 
Duolingo, but all the articles and their translations are manually added by 
Duolingo or users from Duolingo. Therefore, parallel articles in Duolingo are old and 
limited. However, our chrome extension is always working even for those up to the 
minute news and our user can just practice their foreign language in their daily 
readings.

\textbf{Google Translate:} ``Highlight or right-click on a section of text and click
on Translate icon next to it to translate it to your 
language''\footnote{\url{http://en.wikipedia.org/wiki/Google_Chrome_Extensions}}. 
% Tao: please cite
Google Translate is a chrome extension that displays only the translation when user 
select a section, which can be a word, a phrase, a sentence or even a whole page. 
Our chrome extension will translate a single word only, and display the translation,
following with the pronunciations and example sentences to help user understand and 
remember this word. Compared with our extension, Google Translate is more like an extension 
to help user understand the content of the page. Furthermore, our extension will display 
the most appropriate translation as it will refer to the context of the word.

%\begin{table}[ht]
%  \caption{Summary of the differences}
%  \label{table:difference_summary}
%  \begin{center}
%  \begin{tabular}{| p{2.4cm} | p{1.2cm} | p{1.2cm} |  p{1.2cm} |}
%    \hline
%    & Duolingo & Google Translate & Chrome Extension \\
%    \hline
%    Lessons & Yes & No & No \\
%    \hline
%    User's foreign language level & Low & Low-High & Low-High \\
%    \hline
%    Time consuming & Yes & No & No\\
%    \hline
%    Resource & Limited & Infinite & Infinite \\
%    \hline
%    Customizable & Yes & No & Yes \\
%    \hline
%    Link to External Dictionary & No & No & Yes \\
%    \hline
%  \end{tabular}
%  \end{center}
%\end{table}
\begin{figure}[ht]
\centering
    \includegraphics[width=0.9\textwidth]{chrome_extension.jpg}
	\caption{............}
	\label{fig:chrome_extension_1}
\end{figure}
%% Tao: Temporarily put the 3-in-1 screenshot here. Please move it to the proper place.
\begin{figure*}[ht]
\centering
\includegraphics[width=0.99\textwidth]{chrome_extension.jpg}
\caption{Merged screenshots of our Chrome extension on the CNN English
  article {\it Treacherous journey to epicenter of deadly Nepal
    earthquake}.  Underlined components are clickable to yield
  tooltips of two different forms: (l) a definition for learning, (r)
  a multiple-choice interactive test.}
\label{fig:chrome_extension_1}
\end{figure*}

\section{The {\tt SystemA} Chrome Extension}
%%Muthu: To introduce and motivate context here
% Tao: Please mention our bilinguial dictionary 


We give a running scenario to illustrate the use of our language
learning platform, {\tt SystemA}.  When a learner browses to an
English webpage on a news website, our extension selectively replaces
certain original English words words with their Chinese translation
(Figure~\ref{fig:chrome_extension_1}, middle).  While the meaning of
the Chinese word is often apparent in context, the learner can choose
to learn more about the replaced word, by mousing over the translation
to reveal a definition tooltip (Figure~\ref{fig:chrome_extension_1},
left) to aid mastery of the Chinese word.  Once the learner has
encountered the replaced word a few times, {\tt SystemA} will assess
the learner's mastery by generating a multipl choice translation test
on the target word (Figure~\ref{fig:chrome_extension_1}, right).  Our
learning platform thus can be viewed as have three logical components:
{\it translating}, {\it learning} and {\it testing}. \\

%% \begin{figure}[ht]
%%   \centering
%%   \includegraphics[width=0.45\textwidth]{software_design_2.jpg}
%%   \caption{Screen shot of Translating Component}
%%   \label{fig:software_design_2}
%% \end{figure}
{\bf Translating.}  We pass the main content of the webpage from the
extension client to our server for candidate selection and
translation.  As certain words are polysemous, the server must select
the most appropriate translation among all possible meanings.  Our
initial selection method replaces any instance of words stored in our
dictionary.  For translation, we check the word's stored meanings
against the machine translation of each sentence obtained from the
Microsoft Bing Translation API (hereafter, ``Bing'').  Matches are
deemed as correct translations and are pushed back to the Chrome
client for rendering.

%% \begin{figure}[ht]
%%   \centering
%%     \includegraphics[width=0.3\textwidth]{software_design_4.jpg}
%%   \caption{Screen shot of popover with highlighted English word}
%%   \label{fig:software_design_4}
%% \end{figure}
%%  \begin{figure}[ht]
%%      \centering
%%     \includegraphics[width=0.3\textwidth]{software_design_5.jpg}
%%      \caption{Screen shot of popover with highlighted Chinese word}
%%      \label{fig:software_design_5}
%%  \end{figure}

{\bf Learning.} Hovering the mouse over the replacement Chinese word
causes a tooltip to appear, which gives the translation,
pronunciation, simplified and traditional written form, and a {\tt
  More} link that loads additional contextual example sentences (that
were previously translated by the backend) for the learner to study.
The more link must be clicked for activation, as we find this
two-click architecture helps to minimize latency and the loading of
unnecessary data.  The server keeps record of the learning tooltip
activations, logging the enclosing webpage URL, the target word and
the user identity.
% Min: doesn't seem to be shown, actually.  Where is an example sentence?
%  Figure \ref{fig:software_design_5} is the screen
% shot of the pop over with its example sentences.

%% \begin{figure}[ht]
%% \centering
%%   \centering
%%   \includegraphics[width=0.3\textwidth]{software_design_7.jpg}
%%   \caption{Screenshot of English test popover}
%%   \label{fig:software_design_7}
%% \end{figure}
%% \begin{figure}[ht]
%%     \centering
%%   \includegraphics[width=0.3\textwidth]{software_design_8.jpg}
%%   \caption{Screen shot of Chinese test popover}
%%   \label{fig:software_design_8}
%% \end{figure}

{\bf Testing.}  After the learner encounters the same word a
pre-defined number $t=BUG$ times, {\tt SystemA} generates a MCQ test
to assess mastery.  When the learner hovers over the replaced word,
the test is shown for the learner to select the correct answer. When
an option is clicked, the server logs the selection, and the correct
answer is revealed by the client extension.  Statistics on the user's
test history are also updated.

%{\bf Classifying words category.}
% Tao: please keep the label, as I have referred it in wsd section
\subsection{News Categories}
\label{subsec:category}

A key design property of our language learning extension is only
active on certain news websites.  This is important as news articles
typically are classified with respect to a news category, such as
finance, world news, and sports.  If we know which category of news
the learner is viewing, we can leverage this knowledge for improving
the learning experience.

In particular, the category of news can impact  

domain of the 

We propose a simple way to classifying words into different categories from online news articles. To find good “category-related” words, it is essential to get the words from those already classified news articles. The following several steps described the approach we used in classifying words category information. The result of this approach is used as a possible approach in the WSD system (Section~\ref{sec:wds}) and generating suitable distractors (Section~\ref{sec:distractor}). 

{\bf Crawling news content.}
Several web crawlers are designed to get news contents from news websites. The crawler will detect URLs from each news website’s main page as and its sub-category pages. For example, there are sub-categories like “football”, “basketball” under main category “Sports”, and the crawler is able to crawl URLs from “football” page and “basketball” page as well. 

After detailed comparison of most news websites, we divided news articles into seven categories, namely “World”, “Technology”, “Sports”, “Entertainment”, “Finance”, “Health” and “Travel”. Most news articles can be classified into one of the seven categories. The web crawler will store all paragraph tags from each websites and store them as one file under one category. 

{\bf Preprocessing.}
In this step the server uses Natural Language Tool Kit \cite{edw09} for word tokenizing and POS Tagging. The server will store the POS tag of each word. After elimination of all non-English words and those words that contain special symbols, like ``O’Real", ``S\$40", all words that contains only alphabetic letters are conserved. All stop words are also eliminated as well. They are stored as lower case for the ease of future process.

{\bf Classification.}
In the classification step, the server counts the document frequency of each word in all those stored news articles, i.e. if word “scored” appeared 4 times in one article, it will only be counted as once. By following this approach we can successfully reduce the bias of some words only appear a lot of times in one article while don’t appear often in other article. As we are storing similar number of articles in each category, this approach will provide a fair comparison of each word’s popularity among different categories. After this step we will know the document frequency count of each word in different category. 
Assume C is the list of category names, and f(w, C(i))=m means word w appeared in category C(i) for m times, then the sum weight of word w as sw(w) is calculated in Equation~\ref{equation:Distractor_1}:

\begin{equation}
sw (w) = \sum_{i=1}^{n} f(w,C(i))
\label{equation:Distractor_1}
\end{equation}  

A word w is classified into category C(i) if it satisfies Equation~\ref{equation:Distractor_3}::

\begin{equation}
f (w, C(i)) - sw(w)/n >= \delta
\label{equation:Distractor_3} 
\end{equation}  

The confidence factor $\delta$ can be a positive integer between 0 and the average number of articles in each category. It means on average, the word w must appear in a specific category C(i) $\delta$ times more than it appear in other category before it can be classified into category C(i).

It is obvious that a higher confidence factor value will result in less number words getting classified, but it will result in getting words that are more accurate. A suitable confident value is selected to generating category-related words in the later section.






\section{Practical Word Sense Disambiguation}
\label{sec:wds}
\begin{CJK}{UTF8}{gbsn}

As we all know, one word often have multiple translations in another language, and our extension is expected to show the most appropriate one based on the context. We call such translation selection as word sense disambiguation (WSD). WSD is an open task in natural language processing, aiming at identifying the proper sense ({\it i.e.}, meaning) of a word  in a context, when the word has multiple meanings~\cite{Navigli2009}. Traditionally, WSD system identifies the proper sense in the same language, while we show the proper sense in the form of another language.

% Tao: mention the dict in sec2?
In WSD, context information is the key to disambiguate word sense. We, therefore, make use of different granularity of context, {\it i.e.}, the category of the news, the word class, and the sentence, to select proper translations from our bilingual dictionary.


% Tao: Please bold the correct answer. In the caption, indicate the meaning of bold  font. Does the blank means no returned result? Please also indicate it in the caption. Longer and meaningful caption is fine.
                                                         
\begin{table*}[t]
  \caption{Example input and output of our word sense disambiguation configurations. {\bf Boldface} indicates (Column 1) the target word to translate and, (Columns 3--8) the correct translation(s).}
  \label{table:wsd_1}
  \begin{center}
  \begin{tabular}{| p{4cm} | p{3.5cm} | p{1.2cm} | p{1.3cm}| p{0.8cm} | p{0.9cm} | p{1cm} |}
    \hline
    English Sentence & Dictionary & Baseline & Category & Bing & Bing+ & Bing++ \\
    \hline
    ... treating me {\bf like} family ... & \parbox[t]{3cm}{verb : 喜欢, 爱...\\ ... \\preposition : 好像, 好比 ...} & 喜欢 & 好像 & & & \\
    \hline
    ... painting a {\bf picture} of urban street life ... & \parbox[t]{3cm}{... 相, 影, 影片(entertainment), 帧, 想象, 画 ...} & & 影片 & & & \\
    \hline
    ... pistol a {\bf pump} shotgun ... & \parbox[t]{3cm}{verb:抽, 抽水, 打气, 唧, 唧筒, 套\\ noun:抽水机, 唧筒} & & & 唧筒 & & \\
    \hline
    ... have made it into the world's {\bf top} 40 clubs ... & \parbox[t]{3cm}{顶部, 顶端, 顶, 颠, 盖, 极 ...} & 顶部 &  & 顶 & 顶级 & \\
    \hline
    {\bf state} department spokeswoman ... & \parbox[t]{3cm}{...陈, 陈说, 称, 称述, 发表, 发言...} & & & 发言 & 发言人 & 国家 \\
    \hline
    %...  ... &  & \parbox[t]{3cm}{...  ...} & & & & & \\
    %\hline
  \end{tabular}
  \end{center}
\end{table*}

%\subsection{Baseline}
%The simplest way to select a translation from the candidates is by random. However, the correctness of this method is very low, probably less than 20\%, and is not a good baseline for other methods to compete with. Another simple idea is to always select the most commonly used translation. Luckily, when I crawled the dictionary, Google Translate does provide usage frequency of each Chinese Translation.  This turns out to be a much better result, and thus serves as a fair baseline method.


\subsection{News Category}
Topic information have been shown useful in WSD~\cite{Boyd-Graber2007}. Take English word  ``interest" as an example. In finance related articles, ``interest" is more likely to be ``a share, right, or title in the ownership of property" (``利息" in Chinese), than `the feeling of a person whose attention, concern, or curiosity is particularly engaged by something" (``兴趣").  Therefore, analysing the topic of the original article and selecting the translation with the same topic label might help disambiguate the word sense. We leverage the algorithm described in Section~\ref{subsec:category} to obtain the category for news and candidate Chinese translations. 


\subsection{Part-of-Speech Tagger}
The word class, {\it i.e.}, the Part-of-Speech (POS) tag is believed to be beneficial for WSD~\cite{Wilks1998} and Machine Translation~\cite{Toutanova2002,Ueffing2003}.
For example, the English word ``book" has two major classes, verb and noun, meaning ``reserve" (``预定" in Chinese) and ``printed work" (``书"), respectively.
% Tao: stop here.


As we all know, many English words have more than one Part-of-Speech (POS) tags and their Chinese translations in different POS may differ a lot. For example, the word ``book" has two POS tags, noun and verb. If it is used as a noun, mostly it means a handwritten or printed work of fiction or nonfiction, which should be translated as ``书", and mostly means to reserve if used as a verb, which should be translated as ``预定". Therefore, getting the POS tag of the English word might help us identify its sense or the Chinese translation. We decide use Stanford Log-linear Part-of-Speech Tagger \cite{Toutanova2003}.

Firstly, if the word "like" need to be translated, the algorithm will fetch all the Chinese translations as well as their Part-of-Speech tag from our dictionary. Secondly, the algorithm will send the original English sentence to Part-of-Speech Tagger, which is a Java package and has been wrapped into a server. After the client has got the output from the server, it will fetch the corresponding tag and match it to Part-of-Speech tag based on the guidelines mentioned above. Lastly, it will select the translations based on the POS.



\subsection{Machine Translation}
Since our target is to select the most appropriate translation based on the context, using existing Machine Translation (MT) systems is also a good approach, as all of them will certainly translate words based on the context. After I tried a few on-line or off-line MT systems, We decide to use Bing Translator as our Machine Translation system.

{\bf Bing.}
In Table~\ref{table:wsd_1}, the thrid example, the original English sentence is ``including a 45-caliber pistol a pump shotgun and an ar-15 rifle" and ``pump" is the word that we want to translate. Firstly, this algorithm will fetch all the Chinese translations from the database. Next, it will send the original English sentence to Bing Translator using the API provided by Microsoft and get the result that returned from Bing Translator. After that, for each Chinese translation, I will check whether this translation is a substring of the Bing Translator result. If there are a few translations that can match with the Bing Translator result, I will select the longest translation. If there are a few translations with the same length and all of them can match with the Bing Translator result, I will select the translation with the highest frequency of use. In this example, both ``唧" and ``唧筒" are the substrings of Bing Translator result. As ``唧筒" have two characters and ``唧" only have one character, this algorithm will take ``唧筒" as the final result.

{\bf Bing+.}
Bing approach is not perfect. The results that generated by Bing approach is limited by the covearge of our dictionary size. In Table~\ref{table:wsd_1}, the fourth example is the approach of using Bing Translator together with Stanford Word Segmenter, and I would like to use Bing+ to represent this algorithm. The Bing approach will generate ``顶" as the result. After that, our algorithm will send the Chinese sentence returned from Bing Translator to Stanford Word Segmenter. Then, this algorithm will use the segmented word that contains the Bing result as a substring or equals to the Bing result as the final result. In this example, the final result of Bing+ is ``顶级" which is the best result that can be generated from the result of Bing Translator and also a result that does not covered by our dictionary.

{\bf Bing++.}
Bing+ approach is not perfect as well. The results from Bing+ approach is highly related to the accuracy of string matching algorithm. If two English words shares very similar translations or if two Chinese words contains the same Chinese charater, Bing+ approach will generate the wrong result and that's why we need a Word Alignment tool.Bitext word alignment or simply word alignment is the natural language processing task of identifying translation relationships among the words (or more rarely multiword units) in a bitext, resulting in a bipartite graph between the two sides of the bitext, with an arc between two words if and only if they are translations of one another. I use Bing Word Alignment API\footnote{\url{https://msdn.microsoft.com/en-us/library/dn198370.aspx}} as our Word Alignment tool.
The Bing++ algorithm is basically the approach of using Bing+ approach together with the Microsoft Bing Word Alignment. In Table~\ref{table:wsd_1}, the fifth example, ``state" is the word that need to be translated. The result from Bing+ approach is ``发言人", which is the translation of ``spokeswoman", because the Chinese translation ``发言" can be translated from both ``state" and ``spokeswoman". Then step five will send the original English sentence to Bing Word Alignment. Now, there will be two final results, one from Bing+ approach and the other one from Bing Word Alignment and the algorithm will choose the correct one from these two results. In this example, ``state" will match with ``国家" and the algorithm will choose ``国家" as the final result as well.

\subsection{Evaluation}
Our Word Sense Disambiguate System can be evaluated from two important aspects: coverage (i.e., is able to return a translation) and accuracy (i.e., the translation is proper). To this end, I manually annotate the ground truth.

% Tao: The results for news category is very poor. Does it make more sense to combine category with the baseline (frequency-based) method? Say, if category is not matched, we downgrade to frequency-based method.

\begin{table}[ht]
  \caption{Experimental results.}
  \label{table:evaluation_1}
  \begin{tabular}{| p{2cm} | p{2cm} | p{2cm} |}
    \hline
     & Coverage & Accuracy\\
    \hline
    Baseline & 100\% & 57.3\%\\
    \hline
    POSTagger & 94.5\% & 55.2\%\\
    \hline
    News Category & 2.0\% & 7.1\%\\
    \hline
    Bing & 78.5\% & 79.8\%\\
    \hline
    Bing+ & 75.7\% & 80.9\%\\
    \hline
    Bing++ & 76.9\% & 97.4\%\\
    \hline
  \end{tabular}
\end{table}

Table~\ref{table:evaluation_1} column two contains the coverage for different approaches. As the algorithm will try to translate some word only if it is covered by our dictionary, the coverage for Baseline is always 100\%. The coverage for Bing, Bing+, Bing++ and POSTagger are roughly the same and all of them are acceptable. However, the coverage for News Category approach is only 2.0\%. One reason is that when I set the threshold for assigning categories for Chinese word, I purposely make it very high to maximize the accuracy. If the accuracy is quite high, which means this approach is quite useful, then I will lower the threshold and find the balance point.

Figure~\ref{table:evaluation_1} column three contains the accuracy of all the approaches. The last column is the accuracy for News Category approach and it is only 7.1\%. As mentioned in above Chapter, since the accuracy is very low, there is no need to lower the threshold and try to allocate more categories for Chinese words. The accuracy for Baseline is 57.3\%, which is already a fairly hight accuracy. The accuracy for  POSTagger is around 55.2\% also, which is a bit lower than our expectation. The accuracy for Bing++ is 97.4\% which I think is a very good result and it is already very hard to improve. Therefore, based on my test results, Bing++ is the best approach among these five approaches.

% Tao: Could you show some examples that Bing++ failed? And also analyze why.

\end{CJK}

%\section{Distractors Generation Algorithm}
\label{sec:distractor}
Assesing mastery over vocabulary is a key functionality in our extension. In this section, we investigate a way to automatically generate suitable distractors (in English form) for a target word. We postulate ``a  set of suitable distractors'' as: 1) having the same form as the target word, 2) fitting the reading context, and 3) having proper difficulty level according to user's level of mastery.
%By applying a part-of-speech tagger, we obtain the POS tag for the target word, and then restrict the  %candidate distractors to be selected from the same word class.
We obtain the POS tag for the target word and restrict the candidate distractors to the same word class.
To make the distractors fit the context, we identify the target word's news category (approach is detailed in Section~\ref{subsec:category}), and select the distractors from the same category.

%To generate good category-related distractors, it is essential to gather enough words that are more related in a certain category to serve as distractors candidates. By using the approach discussed in Section~\ref{subsec:category}, we crawled more than 1400 articles for seven categories, with around 200 articles in each category. The confidence factor is selected to be 10, which is suitable to classify enough words into different categories. After this step, there should be sufficient “Category-Related" words in each category.


The difficulty of a distractor is measured by its {\bf semantic distance} to the target word: the closer it is to the target word, the more difficult the distractor is. To quantify the semantic distance, we employ  Lin's Distance~\cite{lin98} to measure the distance between two words in WordNet~\cite{Miller1995} and define distractors to be difficult if the Lin's Distance score is below some threshold. 
%By observing the generated distractors, 
We empirically set $0.1$ as the threshold.

%The selection of threshold value will have a direct effect on the speed of distractors generation process. As a very high threshold value will result in more rounds of calculation in semantic distance calculator, and it will take a long time before the distractors are returned to the front end. After several rounds of analysis of each category’s words and the results returned from semantic distance calculator, the threshold value of 0.1 is selected.

%{\bf Semantic Distance.}
%Before we go to explain the next step, it is essential to introduce the semantic distance calculator we used in the server implementation. 
%
%The perspective of semantic relatedness or its inverse, semantic distance, is a concept that indicates the likeness of two words. It is more general than the concept of similarity as stated in WordNet’s synset relation. Similar entities in WordNet are classified into same synset based on their similarity. However, dissimilar entries may also have a close semantic connection by lexical relationships  such as meronymy (car-wheel) and antonymy (hot-cold), or just by any kind of functional relationship or frequent association(pencil-paper, penguin-Antarctica) \cite{ale01}. Semantic distance calculator aims to calculate the semantic relatedness score between two words.
%
%There are many approaches to calculate semantic relatedness score. In this application, we are using Lin Distance \cite{lin98} to calculate the semantic distance between two concepts. The detail of Lin Distance methodology is explained as follows.
%
%Lin attempted to define a measure of semantic similarity that would be both universal and theoretically justified. There are three intuitions that he used as a basis:
%\begin{itemize}
%\item The similarity between arbitrary objects A and B is related to their commonality; the more commonality they share, the more similar they are;
%\item The similarity between A and B is related to the differences between them; the more differences they have, the less similar they are.
%\item The maximum similarity between A and B is reached when A and B are identical, no matter how much commonality they share. 
%\end{itemize}
%
%Based on the intuition above, Lin proposed his approach in measuring similarity between two concepts c1, c2 in Equation~\ref{equation:Distractor_4}:
%
%\begin{equation}
%sim(c1,c2) = \frac{2*log_p(lso(c1,c2))}{log_p(c1)+log_p(c2)}
%\label{equation:Distractor_4}
%\end{equation}  
%
%where p(c) denotes the probability of encountering concept c, and lso(c1,c2) denotes the lowest common subsumer, which is the lowest node in WordNet hierarchy that is a hypernym of c1 and c2. 
%
%The distance calculator will return a score from 0 to 1, as can be easily seen from the formula above. If the score is closer to 1, it means the two words are closer in semantic sense. This distance calculator will play an important role in the following algorithm. 

%\subsubsection{Distractors Selection Algorithm}
\subsection{User knowledge Aware Approach}
As previously mentioned, our extension logs user's detailed learning history. We categorize a user's knowledge on a certain word into three levels, based on the number of times that he / she has encountered  the word. Then we adopt different strategies to generate distractors for users in different knowledge levels. 

{\bf Knowledge Level 1 (K1)}: This is the default knowledge level assigned to a user on a new word. Users aren't tested on words where their knowledge level is K1.
%Considering this, our system prefers to generate simple distractors, and thus randomly select three words %from the same news category. 

{\bf Knowledge Level 2 (K2)}: This indicates that the user has known this word for three times. Therefore, the testing is expected to be harder. The first two distractors are randomly selected from those words that share the same news category. For the third distractor, its semantic distance to the target word is checked in addition, making it a more difficult distractor.  
%For the third distractor, the system keeps randomly selecting distractor from the same category, computing %its semantic distance to the target word, and stops until meeting a difficulty one.

{\bf Knowledge Level 3 (K3)}: At this level the user is expected to have a good understanding of the word since she has {\it i.e.}, passed the test six times. Therefore, we make the test even harder, and choose  all three distractors from the same news category along with the semantic distance criteria. 

%; the algorithm will choose distractors solely based on results returned from semantic distance calculator. Similar to the approach when knowledge level is 2, the algorithm will randomly select word from current category’s word list and calculate the semantic distance between the selected word and the target word. If the score is above certain threshold, the selected word is chosen as one of the distractors. The process is continued until the server can find three distractors. 

\subsection{Evaluation}
% Tao: How to choose the third distractor? from d2's 
% Zhao: The three distractors are all selected from the same synset.
To compare with our proposed method, we reimplemented an existing distractor generation method used in WordGap system~\cite{Knoop2013}. WordGap adopts a knowledge-based approach: selecting the synonyms of synonyms (computed in WordNet) as distractors. That is, they select the most frequently used word, w1, from the target word's synonym set. %Then we select the synset, let's call it  s1, which the synset where the most frequently used word of w1 lies in. 
Then they select the synonyms of w1 and call this set as s1.
Synset s1 contains all the words that are synonyms of synonyms of the target word. Finally they select three most frequently used words from s1 as distractors. This we use is as our baseline approach for comparison.

Our proposed method adopts three different strategies to generate distractors according to user's knowledge level. In our evaluation, we study distractors generated for the two extreme cases, {\it i.e.}, knowledge level 1, and knowledge level 3. Therefore, we conduct a pairwise comparison -- K1  vs. Baseline, and K3  vs. Baseline, using the same test dataset.

%There are two evaluations to be done as follows:
%1.  Compare Baseline with Knowledge Level 1 Algorithm
%
%.  Compare Baseline with Knowledge Level 3 Algorithm
%For each comparison, three distractors are generated from the baseline algorithm; three distractors are generated from the stated algorithm in this report. With the first comparison we will be able to see if the category information will help in selecting more suitable distractors. By comparing the results from the both evaluation, we will be able to see if semantic distance and category information will help improve the suitability of distractors.

%he first distractor {\textit d1} is the most frequently used word from target word’s synonym set, and the second distractor {\textit d2} is the most frequently used synonym for  {\textit d1},
%similarly, {\textit d3} BUG (HOW TO GENERATE d3). However, if the number of valid result we can get is less than three, we will choose the word that shares the same antonym with the target word.

%To evaluate the distractors selection strategy as described in this report, we chose the knowledge-based approach used by many other language learning systems, which is to utilize the WordNet data and selection distractors based on synonyms of synonyms. WordGap system uses this approach to generate vocabulary test for its android application.

%In our implementation of the baseline algorithm, we will choose the most frequent used word w1 from the target word’s synonym set, and select the most frequent used word w2 from word w1’s synonym set. The selection process is continued until we can find 3 distractors to form a vocabulary test. However, if the number of valid result we can get is less than 3, we will choose the word that shares the same antonym with the target word.


\subsubsection{User Study}
To compare the two approaches in generating distractors, we ask users to compare the plausibility of distractors.
%To compare the two approaches in generating distractors, we designed several survey sets to ask users to compare the plausibility of distractors.
 We randomly selected 50 sentences from recent news articles and then chose a noun or adjective from the sentence as the target word. In our survey, each question looks like a real MCQ quiz: we show the original sentence (leaving the target word as blank) as the context, and randomly display the six distractors and the target word as choices. Users are required to read the sentence and select the correct answer (that they think) as rating 1, and rank the other choices from 2 (most plausible) to 7 (least plausible) based on their plausibility. Figure~\ref{fig:distractor_1} shows an example survey question.
 
 %In the survey, participants are required to answer each question and rank the plausibility of all distractors from 1 to 7. The correct answer will be ranked as 1, and the least plausible distractor will be ranked as 7. A screenshot of one sample question is shown in Figure \ref{fig:distractor_1}.

We have two tests (K1  vs. Baseline, and K3  vs. Baseline) and each contains 50 questions. We further group 25 questions as one session, and give users the freedom to participate one or more sessions. Each question will be answered by at least five different users.
Finally, we recruited 15 users from our university, and half of them are native English speakers. 
% Tao: please check
In average, each user participate two sessions.



%The evaluation contains 100 questions and is separated into 4 surveys, with each survey containing 25 questions. Each participant is free to choose one or more than one surveys. The purpose is to reduce the workload in each survey to get better responses. The surveys are sent to Year 1 students from School of Computing, National University of Singapore.  There are 15 valid responses with each participant ranking each distractor with a different weight from 1 to 7. Half of the participants are native English speakers.



\begin{figure}[ht]
   \centering
   \includegraphics[width=0.45\textwidth]{distractor_1.jpg}
   \caption{A sample survey question}
   \label{fig:distractor_1}
\end{figure}

\subsubsection{Results and Analysis}
As each question is answered by five different users, we compute the average rating for each choice. A lower rating means a more plausible (harder) distractor. 
Unsurprisingly, the rating for all the target words is low ($1.1$ in average), as they are the ground truth. This implies that the users answered the survey questions seriously, and the evaluation quality is controlled. For each question, we determine a  algorithm to be the winner if its three distractors as a whole (the sum of three average ratings) are more plausible than the distractors by another algorithm. We calculate the number of winning questions for each algorithm and compute the average score across the 50 questions.  Winning more questions, and obtaining a lower average score denotes a better performance for an algorithm.



%Each participant’s rank will be the weight of the particular distractor in that question, i.e. if the user rank one distractor as rank “5”, the weight of this distractor in this user’s response will be 5. For each distractor of each question, the ranks of all users’ responses are summed. As the more plausible the distractor is, the higher rank it will have, thus if the sum is higher, the approach is not as plausible as the other from user’s point of view.

\begin{table}[th]
    \caption{ Baseline vs. Knowledge Level 1}
    \label{table:distractor_1}
    \begin{center}
    \begin{tabular}{| p{1.5cm} | p{2.5cm} | p{2.2cm} |}
        \hline
         & Number of winning questions & Average score\\
        \hline
        Baseline & 27 & 3.84\\
        \hline
        K1 & 23 & 4.10\\
        \hline
    \end{tabular}
    \end{center}
\end{table}

\begin{table}[th]
    \caption{Baseline vs. Knowledge Level 3 }
    \label{table:distractor_2}
    \begin{center}
    \begin{tabular}{| p{1.5cm} | p{2.5cm} | p{2.2cm} |}
        \hline
         & Number of winning questions & Average score\\
        \hline
        Baseline & 21 & 4.16\\
        \hline
        K3 & 29 & 3.49\\
        \hline
    \end{tabular}
    \end{center}
\end{table}

We display the results for Baseline vs. K1 and Baseline vs. K3, in  
Table \ref{table:distractor_1} and Table \ref{table:distractor_2}, respectively.  We see baseline outperforms the K1 algorithm by four more winning questions and 0.26 average score. Recall that, K1 algorithm is solely relied on category information, without taking word semantic relatedness into account. When a target word does not have a strong category tendency, {\it e.g.}, ``venue" and ``week", it is hard for K1 algorithm to select plausible distractors. On the other hand, we see context information ({\it i.e.}, new category) do play a key role, as K1 wins for 23 times.


In Table~\ref{table:distractor_2}, we see our K3 algorithm significantly betters the baseline for both winning questions (8 more) and average score ($0.67$ lower). This further confirms that context and semantic information are complementary for distractor generation, and a good distractor should fit the reading context and have a certain level of difficulty.
 
 


%If for any question, the sum of weight from all participants for one approach is bigger than the other, then this approach is considered to have won this question. The “average score” is the average sum of weight from each approach for all questions. The lower the average score is, the better performance this approach has gained.

%From Table \ref{table:distractor_1} we can see that in the first comparison, the baseline algorithm actually outscored the knowledge level 1 generation algorithm by 4 questions, with a sum of weight lower than 0.26. From Table \ref{table:distractor_1} we can see that in the second comparison, the knowledge level 3 generation algorithm surpassed the baseline algorithm by 8 questions, with the average weight of 3.49 vs 4.16. 

%\subsubsection{Analysis}
%In knowledge level 1 generation algorithm, there is no semantic distance calculation involved. If the target word to test has no strong category indication, for example, words like 'venue', 'week', it is possible that the knowledge level 1 algorithm will select some distractors that are not as plausible as those coming from the target word's synonym of synonym. 

%However, this problem is solved with the help of semantic distance calculator. In the knowledge level 3 generation algorithm, the distractors chosen are both semantic close and also category-related, which produced a relatively better experiment result.

%Also in the baseline algorithm, it is possible that it will select words that are very rare in real life \cite{sus13}, which may also have influence in the result.

%\section{Platform Viability and Usability Survey}

We have thus far described and evaluated two critical components that
can benefit from capturing the learner's news article context.  In the
larger context, we also need to check the viability of second language
learning intertwined with news reading.  In a requirements survey
prior to the prototype development, two-thirds of the respondents
indicated that although they have used language learning software,
they use it infrequently (less than once per week), giving us motivation for our development.  

Post-prototype, we conducted a summative survey to assess whether our
prototype product satisfied the target niche, in terms of interest,
usability and possible interference with normal reading activities.
We gathered 16 respondents, 15 of which were between the ages of
18--24.  11 (the majority) also claimed native Chinese language
proficiency.  

% Min: Yue, need your input here.  Something like the below.
The respondents felt that the extension platform was a viable language
learning platform (3.4 of 5; on a scale of 1 ``disagreement'' to 
5 ``agreement'') and that they would like to try it when available
for their language pair (3 of 5).

In our original prototype, we replaced the original English word with
the Chinese translation.  While most felt that replacing the original
English with the Chinese translation would not hamper their reading,
they still felt a bit uncomfortable (3.7 of 5).  This finding
prompted us to review and change the default setting of the learning
tooltip to simply add an underline to hint at the tooltip presence.
%\section{Conclusion}
\label{sec:conclusion}

We described WordNews, a client extension and server backend that
transforms the web browser into a second language learning platform.
Leveraging web-based machine translation APIs and a static dictionary,
it offers a viable user-driven language learning experience by pairing
an improved, context-sensitive tooltip definition service with the
generation of context-sensitive multiple choice questions.

WordNews is potentially not confined to use in news websites; one
respondent noted that they would like to use it on arbitrary websites,
but currently we feel usable word sense disambiguation is difficult
enough even in the restricted news domain.  We also note that
respondents are more willing to use a mobile client for news reading,
such that our future development work may be geared towards an
independent mobile application, rather than a browser extension.  We
also plan to conduct a longitudinal study with a cohort of second
language learners to better evaluate WordNews' real-world effectiveness.



% include your own bib file like this:
%\bibliographystyle{acl}
%\bibliography{acl2015}
% use socreport.bst
\bibliographystyle{acl}
\bibliography{acl2015}

%\clearpage\end{CJK*}
\end{document}
